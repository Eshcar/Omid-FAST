
\sys\ provides transactional access to a large collection of persistent data items identified by unique keys. 
The service is persistent and highly available, 
whereas its clients are ephemeral, i.e., they are alive only when performing operations and may fail at any time.
We now discuss the semantics and API \sys\ offers.

{\paragraph{Semantics.} 
A \emph{transaction} is a sequence of put and get operations on different objects that ensures the so-called ACID properties:
\emph{atomicity} (all-or-nothing execution), \emph{consistency} (preserving each object's semantics), 
\emph{isolation} (in that concurrent transactions do not see each other's partial updates), and 
\emph{durability} (whereby updates survive crashes).

Different isolation levels can be considered for the third property. \sys\ opts for  snapshot isolation~\cite{DBLP:conf/sigmod/BerensonBGMOO95}, which is provided by popular database technologies such as Oracle, PostgreSQL, and SQL Server. 
Intuitively, SI ensures that the information a transaction retrieves from the database 
does not mix old and new values. For example, if a task updates the values of two stocks, then no other transaction may observe the old value of one of these stocks and the new value of the other. 
%
More precisely, SI enforces a total order on committed transactions according to their commit times so that 

\begin{enumerate}
    \setlength{\itemsep}{0pt}
    \setlength{\parskip}{0pt}
    \setlength{\parsep}{2pt}  
\item
each transaction's get operations see a consistent snapshot of the database reflecting put operations by
 exactly those transactions that committed prior to the transaction's start time; and 
\item
 a transaction commits only if none of the items it updates has been modified since that snapshot.
 \end{enumerate}
Note that under SI,  concurrent transactions conflict only if they  \emph{update} the same item, whereas with 
serializability, a transaction that updates an item conflicts with transactions that \emph{get} that item. Thus, for read-dominated workloads, SI is amenable to implementations (using multi-versioned concurrency control) that 
allow more concurrency than serializable ones, and hence scale better.

\paragraph{API.} 
\sys's client API offers abstractions both for control (\emph{begin, commit}, and \emph{abort}) and for data access (\emph{get} and \emph{put}). Programmers  delineate transactions via the begin and commit APIs: 
the sequence of get and put operations a client invokes between begin and  commit pertains to one transaction.
Following a commit call, the transaction may successfully \emph{commit}, whereby all of its operations take effect;
in case of conflicts, (i.e., when two concurrent transactions attempt to update the same item), the transaction may
\emph{abort}, in which case none of its changes take effect. An abort may also be initiated by the programmer, e.g., 
on encountering an error. Applications typically retry a transaction upon (either type of) abort. 
\sys's API is offered by a client library, which accesses the data directly in the data store, and interacts with the TM only to begin, commit, or abort transactions. 




