
% Transactions run the world
In recent years, there is an increased focus on supporting large-scale distributed transaction processing;
examples include~\cite{Tephra,Aguilera2015,Balakrishnan2013,Spanner2012,Cowling2012,Dragojevic2015,Kraska2013}.
Transaction systems have many industrial applications, and the need for them is on the rise in the big data world. 
One prominent use case is Internet-scale data processing pipelines, for example, real-time indexing for 
web search~\cite{Percolator2010}. Such systems process information in a streamed fashion, and use shared storage 
in order to facilitate communication between processing stages. Quite often, the different stages process  
data items in parallel, and their execution is subject to data races. Overcoming such race conditions at the application 
level is notoriously complex; the system design is greatly simplified by using the abstraction of transactions with 
well-defined {\em atomicity}, {\em consistency}, {\em isolation}, and {\em durability} (ACID) semantics~\cite{Gray:1992:TPC:573304}. 

% Drum roll - \sys\
We present \sys, an ACID transaction processing system for key-value stores. 
Omid has replaced an initial prototype bearing the same name, 
to which we refer here as Omid1~\cite{OmidICDE2014}, as Yahoo's transaction processing engine; 
it has  been entirely re-designed for scale and reliability, thereby bearing little resemblance with the origin (as discussed 
in Section~\ref{sec:related} below). 
\sys's open source version  recently became an Apache Incubator project\footnote{\url{http://omid.incubator.apache.org}}. 
  
\remove{
Though an initial prototype bearing the same name was presented in the past~\cite{OmidICDE2014}, that prototype was not scalable, as it 
duplicated the entire transaction history at all clients. Moreover, that work did not offer any measures for tolerating server failures.
We present here the current \sys\ system, which has  been entirely re-designed for scale, reliability, and high availability,
thereby bearing little resemblance with the origin. 
}
Internally, \sys\ powers Sieve\footnote{\url{http://yahoohadoop.tumblr.com/post/129089878751}}, 
Yahoo's web-scale content management platform for search and personalization products. Sieve
employs thousands of tasks to digest billions of events per day from a variety of feeds
and push them into a real-time index in a matter of seconds. In this use case, 
tasks need to execute as ACID transactions at a high throughput~\cite{Percolator2010}. 


% Consideration 1: DB-neutrality
The system design has been driven by several important business and deployment considerations. 
First, 
%it was intended for  an environment where persistent key-value stores had already been used. 
guided by the principle of separation of concerns, \sys\ was designed to leverage  
battle-tested key-value store 
technology and support transactions over data stored therein, similar to other industrial 
efforts~\cite{Tephra, Percolator2010, Spanner2012}.
While \sys's design is  compatible with multiple NoSQL key-value stores,
the current implementation 
works with Apache HBase~\cite{hbase}. 

% Consideration 1: simplicity
A second consideration was  simplicity, in order to make the service easy to deploy, support, 
maintain, and monitor in production. This has led to a design based on a centralized \emph{transaction manager} ({\em TM})\footnote{\small{The TM is referred to as Transaction Status Oracle (TSO) in the open source code and documentation.}}.
While its clients and data storage nodes are  widely-distributed and fault-prone, 
\sys's centralized TM provides a single source of truth regarding the transaction history, and facilitates
%lock-free 
conflict resolution among updating transactions (read-only transactions never cause aborts).

% Contributions
Within these constraints, it then became necessary to find novel ways to make the service scalable for throughput-oriented workloads,
and to ensure its continued availability  following failures of clients, storage nodes, and the TM. 
\sys's main contribution is in providing these features:
\begin{description}
\item[Scalability]
\sys\ runs hundreds of thousands of transactions per second over multi-petabyte shared storage.
%A number of aspects of \sys's design contribute to its scalability. First, 
As in other industrial systems~\cite{Percolator2010,OmidICDE2014,Tephra}, scalability is improved by
providing {\em snapshot isolation\/} (SI) rather than serializability~\cite{Gray:1992:TPC:573304} and separating 
data management from control. Additionally, \sys\ employs a unique combination of design choices in the control plane:
(i) synchronization-free transaction processing by a single TM, 
(ii) scale-up of the TM's in-memory conflict detection 
(deciding which transactions may commit) on multi-core hardware, 
and
(iii) scale-out (and scale-up) of metadata update (HBase).
\item[High availability]
The data tier is available by virtue of HBase's reliability, and the TM is implemented as a primary-backup process pair with shared access to critical 
metadata. 
Our solution is unique in 
%allowing fast failover by 
tolerating a potential overlap period when two processes act as primaries,  and
at the same time avoiding costly synchronization (consensus), as long as a single primary is active. 
Note that, being generic, the data tier is not aware of the choice of primary and hence serves operations of both TMs in case of such overlap.
\end{description}

% Roadmap
%The rest of this paper is structured as follows:
We discuss \sys's design considerations in Section~\ref{sec:design}
and related transaction processing systems in Section~\ref{sec:related}.
We detail the system guarantees in Section~\ref{sec:semantics}. Section~\ref{sec:protocol} describes \sys's 
transaction protocol, and Section~\ref{sec:ha} discusses high-availability. An empirical evaluation is given 
in Section~\ref{sec:eval}. We conclude, in Section~\ref{sec:lessons}, by discussing lessons learned from \sys's production 
deployment and our interaction with the open source community, as well as future developments these lessons point to. 
% Finally, Section~\ref{sec:discussion} concludes the paper.
