
\sys\ was originally designed as a foundational building block for Sieve -- Yahoo's next-generation content 
management platform. 
%Sieve ingests new content, and indexes it for search through a variety of processing steps in the real time. 
The need for transactions emerges in scenarios similar to Percolator~\cite{Percolator2010}.  
Analogously to other data pipelines, Sieve is more throughput-sensitive than latency-sensitive. This has led 
to a design that trades off latency for throughput via batching.
%, and scales the service on a single dedicated TM. 
The original design of Omid1~\cite{OmidICDE2014} did not employ a CT, but instead had the TM send clients information
about all pending transactions. This design was abandoned due to limited scalability in the number of clients,
 and was replaced by \sys, which uses the CT to track transaction states. The CT may be sharded for I/O scalability, 
but its update rate is bounded by the resources of the single (albeit multi-threaded) TM; this is mitigated by batching.
%which is a good fit for throughput-oriented applications like Sieve.

%Once \sys\ has been released to open source and became an Apache Incubator project, new use cases began to arise. 
Since becoming an Apache Incubator project, Omid is witnessing increased interest, in a variety of use cases. 
Together with Tephra, it is being considered for use by Apache Phoenix -- an emerging OLTP SQL engine over HBase 
storage~\cite{phoenix}. In that context, latency has increased importance. We are therefore developing a low-latency 
version of \sys\, that has clients update the CT instead of the TM, which eliminates the need for batching and allows 
throughput scaling without sacrificing latency. Similar approaches have been used in Percolator~\cite{Percolator2010}, 
Corfu~\cite{BalakrishnanMDPWW13}, and CockroachDB~\cite{cockroach}. We note, however, that such decentralization 
induces extra synchronization oerhead at commit time and  may increase aborts (in particular, reads may induce aborts); 
the original design may be preferable for throughput-oriented systems.

%A second example, also raised by HortonWorks, was deploying the TM on a non-dedicated machine. Whereas in Sieve each TM runs
%on a dedicated server, there are use cases where \sys\ is just one of many deployed applications, each handling low traffic. 
%This has led us to develop a less CPU-intensive  version of the TM (with no busy waiting for messages) for the open-source project.

Another development %driven by feedback from the open-source community 
is using application semantics to reduce conflict detection. Specifically, some applications can identify scenarios where conflicts 
need not be checked because the use case ensures that they won't happen. Consider, e.g., a massive table load, where records 
are inserted sequentially, hence no conflicts can arise. Another example is a secondary index update, which is guaranteed
to induce no conflict given that the primary table update by the same transaction has been successful. 
To reduce overhead in such cases, we plan to extend the write API to indicate which written keys need
to be tracked for conflict detection. 

%Finally, it may be useful to integrate \sys\ with other persistent multi-versioned key-value stores.%, or use 
%the system to provide ACID transactions along with high-level access abstractions, e.g., a relational database over HBase~\cite{phoenix}. 
On the scalability side, faster technologies may be considered to maintain \sys's commit metadata.  In particular, since \sys's commit table 
is usually written sequentially and infrequently read, it might be more efficient to use log-structured storage that is better optimized for the 
above scenario. Modern hardware (e.g., SSD storage, RDMA networks)  could bring further speedups. 


